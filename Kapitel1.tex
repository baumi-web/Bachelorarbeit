\chapter{Theoretische Grundlagen}

\section{Bedrohungslage}
Die Kernaufgabe eines jeden technischen Sicherheitssystems, insbesondere eines Intrusion Prevention Systems (IPS), ist die Erkennung und Abwehr konkreter Angriffe. Eine systematische Analyse und ein aussagekräftiger Vergleich solcher Systeme, wie er in dieser Arbeit angestrebt wird, setzen daher zwingend ein klares Verständnis der zu bekämpfenden Bedrohungen voraus.\\

 Bevor die Fähigkeiten der Testsysteme praktisch evaluiert werden können, muss dieses theoretische Fundament geschaffen werden. Aus diesem Grund konzentriert sich dieser Abschnitt auf drei fundamentale Angriffskategorien, die für die Prüfung eines IPS zentral sind: Denial-of-Service-Angriffe, die Infiltration durch Malware und die Ausnutzung von Exploits. \\
  Die nachfolgenden Unterkapitel widmen sich jeweils einer dieser Kategorien. Es werden die technischen Grundlagen, gängige Varianten und die charakteristischen Merkmale erläutert, anhand derer ein Schutzsystem den jeweiligen Angriff erkennen kann. Dieses Wissen ist die Voraussetzung, um die Konfiguration der Systeme im Testaufbau nachzuvollziehen und die Ergebnisse der praktischen Tests zu analysieren. \cite{cloudflare1}
  
  
\subsection{Denial-of-Service-Angriff}

Ein Denial-of-Service-Angriff (DoS), oder in seiner heute weitaus verbreiteteren Form als Distributed-Denial-of-Service-Angriff (DDoS), zielt fundamental auf die Beeinträchtigung der Verfügbarkeit eines IT-Dienstes ab. Das Ziel ist die Sabotage des regulären Betriebs, indem die endlichen Ressourcen des Zielsystems gezielt erschöpft werden. Bei einem DDoS-Angriff wird diese Überlastung durch eine Vielzahl kompromittierter Systeme, einem sogenannten Botnetz, gleichzeitig herbeigeführt. Diese massive Parallelisierung macht eine simple Abwehr durch das Blockieren einzelner IP-Adressen praktisch unmöglich.\\

Technisch lassen sich diese Angriffe in drei Hauptkategorien unterteilen, die auf unterschiedliche Schichten des OSI-Modells abzielen:\\

\textbf{Volumetrische Angriffe (Schicht 3/4):} Diese Angriffe zielen darauf ab, die gesamte zur Verfügung stehende Netzwerkbandbreite der Internetanbindung des Ziels zu sättigen. Eine Flut von Paketen wird an ein Ziel gerichtet. Ein klassisches Beispiel ist der UDP-Flood, bei dem eine riesige Menge an UDP-Paketen an zufällige Ports des Zielsystems gesendet wird. Da UDP ein verbindungsloses Protokoll ist, muss der Server für jedes eingehende Paket prüfen, ob eine Anwendung auf dem Port lauscht, und eine ICMP-Antwort "Destination Unreachable" generieren, was seine Ressourcen bindet. Ein ICMP-Flood (oder Ping-Flood) funktioniert nach einem ähnlichen Prinzip, indem er das Ziel mit ICMP-Echo-Request-Paketen überflutet und es zwingt, eine ebenso große Anzahl von Echo-Reply-Paketen zu senden. Das primäre Ziel ist hier die reine Masse an Traffic. \\

\textbf{Protokoll-Angriffe (Schicht 4):} Diese Angriffe nutzen Schwachstellen in der Implementierung von Netzwerkprotokollen aus, um die Ressourcen von Netzwerkkomponenten wie Firewalls oder Load Balancern zu erschöpfen. Der bekannteste Vertreter ist der TCP-SYN-Flood. Er missbraucht den drei-Wege-Handshake von TCP (SYN, SYN-ACK, ACK). Der Angreifer sendet eine hohe Zahl von SYN-Paketen, oft von gefälschten Quell-IP-Adressen. Das Zielsystem antwortet mit SYN-ACK und reserviert Ressourcen in seiner Verbindungstabelle (Backlog Queue) für die erwartete ACK-Antwort. Da diese Antwort niemals eintrifft, bleiben die Verbindungen "halboffen", bis die Tabelle voll ist und keine neuen, legitimen Verbindungsanfragen mehr angenommen werden können und überlastet ist. \\

\textbf{Angriffe auf der Anwendungsebene (Schicht 7):} Diese Angriffe sind subtiler und oft schwerer zu erkennen, da sie scheinbar legitimen Traffic imitieren. Sie zielen nicht auf die Netzwerkbandbreite, sondern auf die CPU- und Speicherressourcen des Servers ab. Beispiele hierfür sind das wiederholte Anfordern sehr großer Dateien oder das Ausführen komplexer Suchanfragen über eine Website, die serverseitig aufwändige Datenbankoperationen auslösen. Auch Angriffe auf Login-Schnittstellen durch massenhafte POST-Requests oder das Ausnutzen rechenintensiver API-Endpunkte fallen in diese Kategorie. Da diese Angriffe oft mit einer relativ geringen Datenrate auskommen, können sie unter dem Radar traditioneller, rein volumetrisch arbeitender Schutzsysteme hindurchgehen.\\ \cite{cloudflare1,rfc1} 

Für ein Intrusion Prevention System besteht die Herausforderung darin, die Muster dieser unterschiedlichen Angriffsarten von legitimen Lastspitzen zu unterscheiden und sie präzise zu blockieren, ohne den regulären Nutzerverkehr zu beeinträchtigen.

\subsection{Malware}

Der Begriff Malware, eine Kurzform für "malicious software", dient als Oberbegriff für jegliche Art von Software, die mit der Absicht entwickelt wurde, auf einem Computersystem unerwünschte oder schädliche Aktionen auszuführen, Daten zu entwenden oder unautorisierten Zugriff zu erlangen. Die Ziele sind dabei vielfältig und reichen vom Diebstahl sensibler persönlicher oder geschäftlicher Informationen über die Störung von Betriebsabläufen, bis hin zur vollständigen Übernahme der Kontrolle über ein System, um es beispielsweise als Teil eines Botnetzes für DDoS-Angriffe zu missbrauchen. Die Infektion eines Netzwerks erfolgt oft mit einem mehrstufigen Prozess.\\

 Zuerst muss die Malware auf das Zielsystem gelangen (Zustellung), was meist durch das Öffnen von manipulierten E-Mail-Anhängen (Phishing), das Klicken auf Links zu kompromittierten Websites (Drive-by-Downloads) oder über infizierte externe Speichermedien geschieht. Anschließend wird die Malware durch eine Benutzerinteraktion oder eine Sicherheitslücke aktiviert (Ausführung) und installiert sich fest im System, um einen Neustart zu überdauern (Persistenz). In vielen Fällen kontaktiert die Schadsoftware danach einen externen Command-and-Control-Server des Angreifers, um Anweisungen zu empfangen oder gestohlene Daten zu übermitteln.\\

Je nach Vorgehensweise und primärer Funktion lässt sich Malware in verschiedene Haupttypen klassifizieren, deren Grenzen jedoch zunehmend verschwimmen. Eine grundlegende Unterscheidung ist für das Verständnis der Bedrohungslandschaft unerlässlich.\\

\textbf{Viren:}\\
Klassische Viren sind Programmsegmente, die sich nicht eigenständig ausführen können, sondern eine legitime, ausführbare Wirtsdatei benötigen, um sich zu verbreiten. Sobald der Nutzer diese infizierte Datei startet, wird auch der virale Code aktiviert, der sich dann in weitere Dateien auf dem System oder auf verbundenen Netzlaufwerken kopiert. Die Schadfunktion, die von der reinen Replikation bis zur Zerstörung von Daten reichen kann, wird erst durch diese Nutzerinteraktion ausgelöst.\cite{BSI1} \\

\textbf{Würmer:}\\
Im Gegensatz dazu sind Würmer eigenständige Schadprogramme, die sich aktiv und ohne Zutun des Nutzers über Netzwerke verbreiten. Sie nutzen gezielt Sicherheitslücken in Betriebssystemen oder Anwendungen aus, um sich von einem infizierten System auf andere, verwundbare Systeme zu replizieren. Dieser autonome Verbreitungsmechanismus kann zu einer extrem schnellen, kaskadenartigen Infektionswelle führen, die ganze Unternehmensnetzwerke lahmlegt. Ein prominentes Beispiel hierfür ist der Wurm WannaCry, der 2017 die ''EternalBlue''-Schwachstelle im SMB-Protokoll von Windows-Systemen ausnutzte, um sich weltweit zu verbreiten und auf den infizierten Systemen Ransomware zu installieren.\cite{BSI12} \\

\textbf{Trojaner:}\\
Trojaner, benannt in Anlehnung an das Trojanische Pferd der griechischen Mythologie, tarnen sich als nützliche oder legitime Software, um den Nutzer zur Installation zu bewegen. Sie enthalten jedoch eine versteckte, bösartige Funktion. Nach der Ausführung installieren sie oft eine Hintertür (Backdoor), die Angreifern einen permanenten und unbemerkten Fernzugriff auf das System ermöglicht. Solche Remote Access Trojans (RATs) erlauben es Angreifern, Daten zu stehlen, das System zu überwachen oder es als Teil eines Botnetzes für weitere Angriffe zu missbrauchen.\cite{kasp1}\\

\textbf{Ransomware:}\\
Eine besonders profitable und schädliche Form ist die Ransomware. Diese Malware verschlüsselt die Dateien des Opfers, sodass auf sie nicht mehr zugegriffen werden kann. Anschließend wird eine Lösegeldforderung angezeigt, meist zahlbar in Kryptowährungen, um die Anonymität der Täter zu wahren. In den letzten Jahren hat sich das Vorgehen zur sogenannten "Double Extortion" (doppelte Erpressung) weiterentwickelt. Hierbei werden die Daten vor der Verschlüsselung zusätzlich vom System des Opfers auf Server der Angreifer kopiert. Wird das Lösegeld nicht gezahlt, drohen die Täter nicht nur mit der permanenten Zerstörung der Daten, sondern auch mit deren Veröffentlichung.\cite{ENISA1}\\

\textbf{Spyware:}\\
Davon abzugrenzen ist Spyware, deren Hauptziel es ist, verdeckt Informationen über den Nutzer, dessen Verhalten und die auf dem System gespeicherten Daten zu sammeln. Eine der bekanntesten Unterarten sind Keylogger, die jeden Tastaturanschlag protokollieren. Auf diese Weise können Angreifer sensible Informationen wie Passwörter, Kreditkartennummern oder private Nachrichten mitschneiden und an ihre Server übermitteln.\\

Durch die Analyse von bekannten Malware-Signaturen, das Erkennen von Anomalien im Protokollverfahren und das Blockieren von Verbindungen zu IP-Adressen und Domains können IPS-Systeme diese Bedrohungen erkennen und unterbinden.



\subsection{Exploits}
Der Begriff Exploit bezeichnet ein spezifisches Software-Fragment, einen Datenblock oder eine Befehlssequenz, die gezielt einen Programmierfehler oder eine konzeptionelle Schwachstelle in einem Computersystem oder einer Anwendung ausnutzt. Es ist entscheidend, zwischen der Schwachstelle, bei der es sich um einen passiven, latenten Fehler im Code handelt, und dem Exploit, dem aktiven Werkzeug zur Ausnutzung dieses Fehlers, zu unterscheiden. Das Ziel eines Exploits ist es, erhöhte Rechte zu erlangen (Privilege Escalation), beliebigen Code auf dem Zielsystem auszuführen (Arbitrary Code Execution) oder einen Denial-of-Service-Zustand auszulösen. Dies macht Exploits zu einem primären Vektor für den erstmaligen Einbruch in ein Netzwerk.\cite{wik12}\\

Der Prozess eines Angriffs, der auf der Ausnutzung von Schwachstellen basiert, lässt sich in die sieben Phasen der Cyber Kill Chain unterteilen, die den Ablauf aus der Perspektive des Angreifers strukturieren. Die erste Phase ist die \textbf{Reconnaissance}, in der ein Angreifer Informationen über das Ziel sammelt, um potenzielle Angriffsvektoren zu finden. In der zweiten Phase, der \textbf{Weaponization}, wird ein passender Exploit-Code mit einer Nutzlast (Payload) – dem eigentlichen Schadcode – gekoppelt. Anschließend erfolgt die \textbf{Delivery}, bei der der vorbereitete Exploit an das Zielsystem übermittelt wird, beispielsweise über eine Phishing-Mail oder einen verwundbaren Netzwerkdienst.\\

In der vierten Phase, der \textbf{Exploitation}, wird der Exploit-Code zur Ausführung gebracht und löst den Fehler in der verwundbaren Software aus. Dies ermöglicht die Ausführung des mitgelieferten Payload. Darauf folgt die \textbf{Installation}, in der die Payload Persistenz auf dem System etabliert, um nach einem Neustart noch auf dem Zielsystem vorhanden zu sein. In der sechsten Phase, \textbf{Command and Control} (C2), baut die installierte Schadsoftware eine ausgehende Verbindung zu einem vom Angreifer kontrollierten Server auf. Über diesen Kanal kann der Angreifer das System fernsteuern. In der letzten Phase, \textbf{Actions on Objectives}, verfolgt der Angreifer seine eigentlichen Ziele, wie zum Beispiel den Diebstahl von Daten, die Ausbreitung im internen Netzwerk, oder die Verschlüsselung von Systemen zur Erpressung von Lösegeld.\cite{wik1}\\

\textbf{Zero-Day-Exploit:}\\
Die gefährlichste Kategorie sind Zero-Day-Exploits. Der Begriff „Zero-Day“ leitet sich aus der Perspektive des Softwareherstellers ab: Ab dem Moment, in dem ein Exploit für eine bisher unbekannte Schwachstelle aktiv ausgenutzt wird, hat der Hersteller null Tage Zeit gehabt, einen entsprechenden Sicherheitspatch zu entwickeln und bereitzustellen. Diese Unkenntnis aufseiten der Verteidiger verschafft den Angreifern ein kritisches Zeitfenster, in dem ihre Angriffe mit sehr hoher Wahrscheinlichkeit erfolgreich sind.\\
Die besondere Gefahr von Zero-Day-Exploits liegt in ihrer Fähigkeit, traditionelle, signaturbasierte Schutzmechanismen wie die meisten Antivirenprogramme und Intrusion Prevention Systeme vollständig zu umgehen. Da der Angriff neu und unbekannt ist, existiert keine Signatur, kein Muster und kein Hashwert, nach dem ein solches System suchen könnte. Der Exploit ist für die Verteidigungslinie demnach quasi unsichtbar.\\
Die Abwehr von Zero-Day-Angriffen erfordert daher fortschrittlichere Sicherheitsstrategien, die über die reine Signaturerkennung hinausgehen. Dazu gehören verhaltensbasierte Analyse (Heuristik) und Anomalieerkennung, bei denen ein System nicht nach bekannten Mustern, sondern nach ungewöhnlichem und potenziell bösartigem Verhalten sucht. Eine weitere wichtige Technik ist das Sandboxing, bei dem verdächtiger Code in einer isolierten, virtuellen Umgebung ausgeführt wird, um seine Aktionen zu beobachten, ohne das eigentliche System zu gefährden.

\section{IDS / IPS Grundlagen}
Nach der detaillierten Betrachtung der vielfältigen Bedrohungslandschaft im vorherigen Kapitel, werden nun die Systeme näher beleuchtet, die entwickelt wurden, um Netzwerke vor genannten Angriffen zu schützen. Während klassische Firewalls den Verkehr primär anhand von Adressen und Ports (Schicht 3 und 4 des OSI-Modells) filtern, gehen moderne Schutzmechanismen einen entscheidenden Schritt weiter, indem sie den Inhalt des Datenverkehrs analysieren. Im Zentrum dieser erweiterten Sicherheitsarchitektur stehen Intrusion Detection- und Intrusion Prevention-Systeme (IDS/IPS).

In den folgenden Absätzen werden dazu die theoretischen Grundlagen dieser Technologien erklärt. Es wird die evolutionäre Entwicklung vom rein passiven Erkennen eines Angriffs (Detection) hin zum aktiven Verhindern (Prevention) nachgezeichnet, um die grundlegenden Unterschiede und die Motivation hinter der Entwicklung von IPS zu beleuchten. Anschließend wird das technische Funktionsprinzip der Deep Packet Inspection (DPI) erläutert, das es diesen Systemen überhaupt erst ermöglicht, den Inhalt von Datenpaketen zu analysieren. Abschließend folgt die Vorstellung der verschiedenen Erkennungsmethoden, auf deren Basis ein System die Entscheidung trifft, ob es sich um legitimen oder bösartigen Datenverkehr handelt.

\subsection{Vom passiven Detektieren zum aktiven Verhindern}

Die historischen Vorläufer moderner Angriffserkennungssysteme sind die Intrusion Detection Systeme (IDS). Sie entstanden aus der Notwendigkeit heraus, die Sicherheitsarchitekturen von Netzwerken zu ergänzen, da klassische Paketfilter-Firewalls allein nicht mehr ausreichten, um Angriffe auf höheren Protokollebenen zu erkennen. Das grundlegende Funktionsprinzip eines netzwerkbasierten IDS (NIDS) ist die der passiven Überwachung. Ein IDS wird im Netzwerk so implementiert, dass es eine Kopie des gesamten zu überwachenden Datenverkehrs erhält, ohne selbst Teil des aktiven Datenpfades zu sein. Technisch wird dies meist über einen sogenannten Mirror-Port (auch SPAN-Port genannt) an einem Netzwerk-Switch realisiert, der den gesamten Verkehr eines oder mehrerer anderer Ports auf den Port des IDS spiegelt. Eine alternative, oft als zuverlässiger geltende Methode ist der Einsatz eines dedizierten Network Taps, einer Hardware-Komponente, die direkt in die Leitung eingeschleift wird und eine verlustfreie Kopie des Verkehrs auskoppelt. Unabhängig von der Methode agiert das IDS somit stets wie ein Beobachter, der den Verkehr analysiert, ihn jedoch nicht direkt beeinflusst. Stellt das System eine potenzielle Bedrohung fest, ist seine primäre Aufgabe das Auslösen eines Alarms. Dieser Alarm kann in Form eines Eintrags in einer Log-Datei, einer Benachrichtigung per E-Mail oder einer Meldung an ein übergeordnetes Security Information and Event Management (SIEM) System erfolgen, wo die Daten für eine weitere Analyse korreliert werden. Die Kernfunktion eines IDS ist also die reine Detektion; es beantwortet die Frage: „Passiert hier gerade etwas Bösartiges?“ \cite{Claudia1}.\\

\textbf{Definition Network Tap:}\\
Ein Network TAP (Test Access Point) ist eine dedizierte Hardware-Komponente, die eine exakte, verlustfreie Kopie des gesamten Datenverkehrs von einer physischen Netzwerkverbindung erstellt. Das Gerät wird direkt in die Kabelverbindung, beispielsweise zwischen einem Router und einem Switch, eingeschleift und leitet den durchfließenden Verkehr passiv an einen oder mehrere Monitor-Ports weiter, an denen das Überwachungssystem angeschlossen ist. Im Gegensatz zu einem SPAN-Port eines Switches, der bei hoher Netzwerkauslastung unter Umständen Pakete verwirft, garantiert ein TAP die Weiterleitung allen Traffics, einschließlich fehlerhafter Pakete. Zudem sind die meisten TAPs ausfallsicher konzipiert.\cite{TAP1}\\

Die entscheidende Schwäche dieses passiven Ansatzes liegt in der systembedingten Latenz zwischen der Erkennung eines Angriffs und der Einleitung einer Gegenmaßnahme. Nachdem ein IDS einen Alarm ausgelöst hat, ist eine Reaktion erforderlich, um den Angriff zu stoppen. Diese Reaktion kann manuell durch einen Administrator erfolgen, der beispielsweise eine neue Regel in der Firewall konfiguriert, oder durch ein nachgeschaltetes, automatisiertes System. In der Zeit, die dieser Prozess unweigerlich in Anspruch nimmt, kann der Angriff sein Ziel bereits erreicht und erheblichen Schaden verursacht haben. Ein weiteres gravierendes, operatives Problem ist die sogenannte „Alert Fatigue“ (Alarm-Müdigkeit). Hochsensibel konfigurierte IDS können eine enorme Menge an Alarmen generieren, von denen viele Falschmeldungen (False Positives) sind. Administratoren werden von dieser Flut an Informationen überfordert, was dazu führen kann, dass sie kritische, echte Alarme übersehen oder weniger ernst nehmen.\\

Um diese kritische Schutzlücke zu schließen und die Reaktionszeit auf null zu reduzieren, wurden die Intrusion Prevention Systeme (IPS) entwickelt. Im Gegensatz zu einem IDS wird ein IPS aktiv und „in-line“ im Netzwerk platziert und behebt damit dessen grundlegende Schwäche. Diese architektonische Entscheidung ist fundamental: Das IPS fungiert als aktiver Kontrollpunkt, den der gesamte Datenverkehr passieren muss, um sein Ziel zu erreichen. Dies birgt jedoch auch neue Risiken: Ein IPS kann zu einem Leistungsengpass (Bottleneck) werden und stellt einen Single Point of Failure dar. Fällt das IPS aus, wird die gesamte Netzwerkverbindung unterbrochen, sofern keine Bypass-Mechanismen vorhanden sind.\\

Diese direkte Positionierung im Datenstrom ermöglicht es einem IPS, nicht nur zu erkennen, sondern auch unmittelbar und automatisiert zu handeln. Erkennt es einen Angriff, kann es eine Reihe von vordefinierten Aktionen durchführen. Die Palette dieser Reaktionsmöglichkeiten ist breit und reicht von dem einfachen Blockieren oder Verwerfen (Block/Drop) einzelner bösartiger Datenpakete über das aktive Zurücksetzen von TCP-Verbindungen (Reset), um eine Sitzung sauber zu beenden, bis hin zur reinen Alarmierung (Alert). Letztere wird genutzt, um neue Regeln im Überwachungsmodus zu testen, ohne den produktiven Verkehr zu gefährden. \cite{Suricata1, NIST1}\\

Darüber hinaus bieten viele Systeme erweiterte, oftmals herstellerabhängige Reaktionsmöglichkeiten. Dazu zählt beispielsweise die temporäre Sperrung der angreifenden IP-Adresse (Shun) oder die Umleitung des Angriffsverkehrs auf ein Ködersystem (Quarantäne/Sinkhole) zur weiteren Analyse und zur Sammlung von Informationen über den Angreifer. Die genaue Auswahl und die Granularität der verfügbaren Aktionen sind ein wesentliches Unterscheidungsmerkmal des jeweiligen Produkts und ein Indikator für seine technische Reife. Welche dieser Funktionen von den in dieser Arbeit untersuchten Systemen, FortiGate und OPNsense, im Detail unterstützt werden, wird im Rahmen ihrer jeweiligen Vorstellung in Kapitel 2.3 detailliert erläutert.
\newpage
\subsection{Funktionsprinzip: Deep Packet Inspection (DPI)}
Die technologische Grundlage, die moderne Intrusion Prevention Systeme von klassischen Firewalls unterscheidet und ihre Effektivität überhaupt erst ermöglicht, ist die Deep Packet Inspection (DPI). Während traditionelle Paketfilter ihre Entscheidungen ausschließlich auf Basis der Header-Informationen der Schichten 3 und 4 des OSI-Modells (IP-Adressen, Ports) treffen, geht die DPI-Technologie entscheidende Schritte weiter. Selbst die fortgeschrittenere Stateful Packet Inspection (SPI), die den Zustand von Verbindungen verfolgt, analysiert nicht den eigentlichen Inhalt der übertragenen Daten. Die DPI hingegen untersucht nicht nur die Header, sondern auch die Nutzlast (Payload) der Datenpakete und kann den Datenstrom bis zur Anwendungsschicht (Schicht 7) analysieren, um so den Kontext und die Bedeutung der übertragenen Daten zu verstehen.\\

Der Funktionsprozess einer DPI-Engine ist hochkomplex und muss in Echtzeit ablaufen. Der erste Schritt in diesem Prozess ist die Paket-Normalisierung und Strom-Reassemblierung. Ein IPS empfängt den Netzwerkverkehr als eine Sequenz einzelner Datenpakete, die durch Fragmentierung oder durch die Netzwerkübertragung selbst in falscher Reihenfolge (Out-of-Order) eintreffen können. Die DPI-Engine muss diese Pakete daher anhand ihrer TCP-Sequenznummern korrekt sortieren und zu einem kohärenten, bidirektionalen Datenstrom zusammensetzen. Dieser Schritt ist kritisch, da viele Angriffstechniken gezielt versuchen, durch manipulierte Fragmentierung die Erkennung zu umgehen. Darauf aufbauend muss die Engine den reassemblierten Datenstrom dekodieren, um die genutzten Anwendungsprotokolle wie HTTP, SMB oder DNS zu identifizieren. Anschließend wird der Strom gemäß der Spezifikation des jeweiligen Protokolls geparst, also in seine logischen Bestandteile zerlegt, um zwischen Kontrollinformationen und der eigentlichen Nutzlast zu unterscheiden.\\

Im Kern der DPI steht dann die eigentliche Inhaltsanalyse. Nachdem die Daten durch die vorherigen Schritte aufbereitet und in ihren protokollspezifischen Kontext gebracht wurden, werden sie an die eigentliche Analyse-Engine des IPS weitergeleitet. Die Aufgabe dieser Engine ist es, den Inhalt der Nutzlast zu bewerten und eine Entscheidung über dessen Zulässigkeit zu treffen. Hierfür greift das System auf verschiedene logische Verfahren zurück, um bösartige von gutartigen Daten zu unterscheiden. Die genaue Funktionsweise dieser zentralen Erkennungsmethoden, insbesondere der signaturbasierten und der anomaliebasierten Analyse, wird im nachfolgenden Kapitel 2.2.3 detailliert erläutert.\\

Die größte technische Hürde stellt jedoch der zunehmend verschlüsselte Datenverkehr (SSL/TLS) dar. Um die Nutzlast von HTTPS-Verbindungen zu inspizieren, muss das IPS eine als SSL/TLS-Inspektion bekannte Methode anwenden und als autorisierter „Man-in-the-Middle“ agieren. Dabei fängt das IPS die Verbindungsanfrage des Clients ab, baut selbst eine Verbindung zum Zielserver auf und präsentiert dem Client ein eigenes zur Laufzeit generiertes Zertifikat. Damit dieser Prozess funktioniert, muss das Zertifikat der IPS-internen Certifikatsstelle auf allen Client-Systemen als vertrauenswürdig hinterlegt sein. Nur so kann das IPS den Verkehr transparent entschlüsseln, mit den beschriebenen Methoden analysieren und anschließend neu verschlüsselt zum Zielserver weiterleiten. Dieser Vorgang ist extrem rechenintensiv und verdeutlicht die Notwendigkeit spezialisierter Hardware in leistungsfähigen IPS-Systemen. Ohne die Fähigkeit der DPI, tief in die Nutzlast von Datenpaketen zu blicken, wäre ein Schutzsystem blind für nahezu alle modernen Angriffe auf der Anwendungsebene. \cite{NIST1,Claudia1}

\subsection{Zentrale Erkennungsmethoden}
Nachdem die Deep Packet Inspection (DPI) den Netzwerkverkehr in einen analysierbaren, reassemblierten Datenstrom umgewandelt hat, beginnt der eigentliche Analyseprozess. An dieser Stelle kommen die Erkennungs-Engines des IPS zum Einsatz, die auf zwei fundamentalen, konzeptionell unterschiedlichen Paradigmen basieren: der signaturbasierten Erkennung, die auch als Missbrauchserkennung (Misuse Detection) bekannt ist, und der anomaliebasierten Erkennung (Anomaly Detection). Diese beiden Methoden bilden das logische Herzstück eines jeden IPS.\\

\textbf{Signaturbasierte Erkennung:}

Die signaturbasierte Erkennung, auch als Missbrauchserkennung (Misuse Detection) bekannt, stellt den fundamentalen und historisch älteren Ansatz zur Identifizierung von Cyber-Angriffen dar. Sie basiert auf einem fest definierten, vorab erstellten Wissensschatz über bekannte Bedrohungen und sucht im Netzwerkverkehr exakt nach den Mustern, die diesen Bedrohungen zugeordnet sind. Das zugrundeliegende Prinzip ist vergleichbar mit dem eines Virenscanners, der nach den eindeutigen "Fingerabdrücken" bekannter Viren sucht, jedoch auf der Ebene von Netzwerkprotokollen und Datenströmen.\\

Der Lebenszyklus einer Signatur beginnt mit der Entdeckung einer neuen Bedrohung. Sicherheitsforscher bei Herstellern wie Fortinet oder in Open-Source-Communities analysieren neue Malware oder die Vorgehensweise eines Exploits im Detail. Dabei extrahieren sie eindeutige, unveränderliche Merkmale. Dies kann eine bestimmte Zeichenfolge in der Nutzlast, eine charakteristische Abfolge von Befehlen oder ein spezifischer Wert in einem Protokoll-Header sein. Auf Basis dieser Analyse wird eine Signatur als formale Regel erstellt. Eine moderne Signatur ist dabei weit mehr als eine simple Zeichenkette. Sie ist ein mehrdimensionales Konstrukt, das typischerweise folgende Elemente umfasst:\\
\setlist{noitemsep}
\begin{itemize}
	\item \textbf{Header-Bedingungen:} \\Diese legen den Geltungsbereich der Regel fest und filtern den zu untersuchenden Verkehr vor. Hierzu gehören Protokoll (TCP, UDP, ICMP), Quell- und Ziel-IP-Adressen (oder ganze Subnetze) sowie Port-Nummern.\\
	\item \textbf{Inhalts-Muster (Payload Content):} \\Der Kern der Signatur. Hier werden die eigentlichen bösartigen Muster definiert, oft als Byte-Sequenzen oder als flexible reguläre Ausdrücke.\\
	\item \textbf{Kontextuelle Optionen: } \\Diese verfeinern die Regel, um Falschmeldungen zu minimieren. So kann eine Regel beispielsweise festlegen, dass ein bestimmtes Muster nur dann als bösartig gilt, wenn es in einem HTTP-POST-Request an einen bestimmten URL-Pfad auftritt oder wenn es nicht von einer bestimmten anderen Zeichenfolge gefolgt wird.\\
	\item \textbf{Metadaten:} \\Jede Signatur wird mit zusätzlichen Informationen wie einer eindeutigen ID, einer Referenz zur entsprechenden CVE-Nummer (Common Vulnerabilities and Exposures), einer Beschreibung des Angriffs und einer Klassifizierung der Bedrohungsschwere versehen
\end{itemize}

Technisch werden zwei Arten von Signaturen unterschieden: atomare Signaturen, die ein einzelnes Paket auf ein Muster hin untersuchen, und zustandsbehaftete (stateful) Signaturen, die den Zustand einer gesamten Verbindung über mehrere Pakete hinweg verfolgen und erst dann anschlagen, wenn eine bestimmte Abfolge von Ereignissen eingetreten ist. Zustandsbehaftete Signaturen sind weitaus mächtiger, aber auch rechenintensiver. Um die immense Aufgabe zu bewältigen, Tausende dieser komplexen Regeln in Echtzeit auf einen Datenstrom mit Gigabit-Geschwindigkeit anzuwenden, werden die Signaturen beim Start des IPS in optimierte Datenstrukturen geladen. Dies ermöglicht einen simultanen Abgleich aller Muster in einem einzigen Durchlauf.\\

Der entscheidende Vorteil der signaturbasierten Erkennung liegt in ihrer hohen Präzision und Zuverlässigkeit. Ein positiver Treffer ist ein starker Indikator für einen tatsächlichen, bekannten Angriff, was die Rate an False Positives extrem niedrig hält. Die Nachteile sind jedoch ebenso nicht zu vernachlässigen. Die Methode ist per Definition reaktiv; sie kann einen Angriff erst erkennen, nachdem er analysiert und eine Signatur dafür erstellt und verteilt wurde. Gegen Zero-Day-Exploits ist sie daher wirkungslos. Zudem versuchen Angreifer gezielt, Signaturen durch Evasion-Techniken zu umgehen. Mittels Polymorphismus oder Metamorphismus wird der Schadcode bei jeder Infektion automatisch so verändert, dass sein "Fingerabdruck" variiert. Durch Verschleierung, beispielsweise durch Kodierung oder Verschlüsselung der Nutzlast, wird versucht, den bösartigen Inhalt vor der DPI-Engine zu verbergen. Die Effektivität eines signaturbasierten IPS hängt somit direkt und unabdingbar von der Qualität, der Geschwindigkeit und der Intelligenz der Signatur-Updates des jeweiligen Herstellers oder der Community ab.\\

\textbf{Anomaliebasierte Erkennung:}

Die anomaliebasierte Erkennung verfolgt einen fundamental anderen, induktiven Ansatz. Statt nach bekannten Mustern des ''Bösen'' zu suchen, versucht sie, ein umfassendes Modell des ''Normalen'' zu erstellen und jede signifikante Abweichung von diesem Zustand als potenzielle Bedrohung zu identifizieren. Dieser Ansatz hat den theoretischen Hauptvorteil, auch bisher unbekannte Zero-Day-Angriffe, neue Malware-Varianten oder gezielte, individuelle Angriffe erkennen zu können, für die keine Signaturen existieren. Die zugrundeliegende Annahme ist, dass jede bösartige Aktivität zwangsläufig das normale Verhalten eines Systems oder Netzwerks stört und messbare Spuren hinterlässt.\\

Der Kernprozess ist die Erstellung und Pflege einer Baseline, eines digitalen Abbilds des Normalzustands. Dieser Prozess beginnt mit einer initialen Trainings- oder Lernphase, in der das System den Netzwerkverkehr über einen längeren Zeitraum passiv beobachtet. In dieser Phase kommen je nach Komplexität des Systems, unterschiedliche Verfahren zum Einsatz, von einfachen statistischen Mittelwerten bis hin zu komplexen Machine-Learning-Modellen. Die erfassten Parameter umfassen eine Vielzahl von Metriken, wie zum Beispiel die durchschnittliche und maximale Bandbreitennutzung pro Protokoll, die Latenzzeiten zu bestimmten Servern, die Größe von Paketen und die typischen Kommunikationsbeziehungen zwischen den Hosts im Netzwerk.\\

Nach der Trainingsphase geht das System in den aktiven Erkennungsmodus über und vergleicht den Echtzeit-Verkehr kontinuierlich mit der erlernten Baseline. Anomalien können dabei auf verschiedenen Ebenen festgestellt werden:\\
\setlist{noitemsep}
\begin{itemize}
	\item \textbf{Protokoll-Anomalien:} \\Dies ist die verlässlichste Form der Anomalieerkennung. Hierbei wird der Datenverkehr von einem strikten Protokoll-Parser analysiert, der auf den offiziellen RFC-Spezifikationen basiert. Jede Abweichung von der Norm – sei es ein ungültiges Flag im TCP-Header, eine fehlerhafte Zeichenkodierung in einer DNS-Anfrage oder die Verwendung eines veralteten Befehls in einer SMB-Sitzung – wird als Anomalie gemeldet. Viele Exploits verursachen solche subtilen Protokollverstöße bei dem Versuch, eine Schwachstelle auszunutzen.\\

\item\textbf{ Statistische Anomalien:} \\Hierbei werden die aktuellen Metriken mit dem statistischen Modell der Baseline verglichen. Das System berechnet für beobachtete Ereignisse einen Anomalie-Score. Ein plötzlicher Anstieg der fehlgeschlagenen Login-Versuche auf einem Server, ein drastischer Anstieg des ausgehenden Verkehrs von einer Workstation mitten in der Nacht oder die Nutzung eines seltenen Protokolls durch einen Benutzer, der dies noch nie getan hat, würde den Score in die Höhe treiben. Überschreitet dieser Score einen vordefinierten Schwellenwert (Threshold), wird ein Alarm ausgelöst.
\end{itemize}
Die größte Herausforderung und der entscheidende Nachteil dieses Ansatzes ist die notorisch hohe Rate an Falschmeldungen (False Positives). Das ''normale'' Verhalten eines Netzwerks ist nicht statisch. Jede legitime, aber neue und unvorhergesehene Aktivität , wie beispielsweise die Einführung eines neuen Cloud-Dienstes, eine Systemwartung oder eine Marketing-Kampagne mit hohem Traffic, kann fälschlicherweise als Anomalie eingestuft werden. Dies führt zum operativen Problem der ''Alert Fatigue'', bei dem Administratoren durch eine Flut an Alarmen die Übersicht verlieren und die Sensitivität des Systems herunterregulieren, wodurch dessen Effektivität sinkt. Zudem besteht die Gefahr der ''Baseline Pollution''. Dies definiert die Lernphase in einem bereits kompromittierten Netzwerk, wodurch das bösartige Verhalten als Teil der Norm erlernt und zukünftig nicht mehr erkannt wird.\\

Aus diesen Gründen wird die anomaliebasierte Erkennung in der Praxis fast immer in einem hybriden Modell als Ergänzung zur signaturbasierten Erkennung eingesetzt. Sie dient als intelligentes Frühwarnsystem für neuartige Bedrohungen.
\section{Vorstellung der Systeme}
Nachdem die vorangegangenen Kapitel die theoretischen Grundlagen der Bedrohungslage und Funktionsweise von Intrusion Prevention Systemen erläutert haben, widmet sich dieses Kapitel der Vorstellung der beiden für den praktischen Vergleich ausgewählten Systeme. Die Auswahl fiel auf eine FortiGate-Firewall als prominenten Vertreter der kommerziellen NGFW und auf OPNsense als repräsentatives Beispiel für eine flexible, weit verbreitete Open-Source-Alternative.\\

Diese beiden Systeme wurden gezielt gewählt, da sie in ihren jeweiligen Segmenten eine hohe Marktrelevanz besitzen und fundamental unterschiedliche Philosophien in Bezug auf Architektur, Lizenzmodell und Verwaltung verkörpern. Um eine fundierte Vergleichsbasis zu schaffen, werden die beiden Systeme zunächst in separaten Unterkapiteln detailliert vorgestellt. Dabei wird auf ihre jeweilige Architektur, die technische Implementierung der IPS-Funktionalität und das zugrundeliegende Betriebsmodell eingegangen. Abschließend fasst eine synoptische Gegenüberstellung die zentralen Unterschiede tabellarisch zusammen, um eine klare Ausgangslage für die Konzeption der Testumgebung im nachfolgenden Kapitel zu schaffen.
\subsection{Der kommerzielle Vertreter: FortiGate}
\subsection{Die Open-Source-Alternative: OPNsense}
\subsection{Synoptische Gegenüberstellung}