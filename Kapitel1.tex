\chapter{Theoretische Grundlagen}

\section{Bedrohungslage}
Die Kernaufgabe eines jeden technischen Sicherheitssystems, insbesondere eines Intrusion Prevention Systems (IPS), ist die Erkennung und Abwehr konkreter Angriffe. Eine systematische Analyse und ein aussagekräftiger Vergleich solcher Systeme, wie er in dieser Arbeit angestrebt wird, setzen daher zwingend ein klares Verständnis der zu bekämpfenden Bedrohungen voraus.\\

 Bevor die Fähigkeiten der Testsysteme praktisch evaluiert werden können, muss dieses theoretische Fundament geschaffen werden. Aus diesem Grund konzentriert sich dieser Abschnitt auf drei fundamentale Angriffskategorien, die für die Prüfung eines IPS zentral sind: Denial-of-Service-Angriffe, die Infiltration durch Malware und die Ausnutzung von Exploits. \\
  Die nachfolgenden Unterkapitel widmen sich jeweils einer dieser Kategorien. Es werden die technischen Grundlagen, gängige Varianten und die charakteristischen Merkmale erläutert, anhand derer ein Schutzsystem den jeweiligen Angriff erkennen kann. Dieses Wissen ist die Voraussetzung, um die Konfiguration der Systeme im Testaufbau nachzuvollziehen und die Ergebnisse der praktischen Tests zu analysieren. \cite{cloudflare1}
  
  
\subsection{Denial-of-Service-Angriff}

Ein Denial-of-Service-Angriff (DoS), oder in seiner heute weitaus verbreiteteren Form als Distributed-Denial-of-Service-Angriff (DDoS), zielt fundamental auf die Beeinträchtigung der Verfügbarkeit eines IT-Dienstes ab. Das Ziel ist die Sabotage des regulären Betriebs, indem die endlichen Ressourcen des Zielsystems gezielt erschöpft werden. Bei einem DDoS-Angriff wird diese Überlastung durch eine Vielzahl kompromittierter Systeme, einem sogenannten Botnetz, gleichzeitig herbeigeführt. Diese massive Parallelisierung macht eine simple Abwehr durch das Blockieren einzelner IP-Adressen praktisch unmöglich.\\

Technisch lassen sich diese Angriffe in drei Hauptkategorien unterteilen, die auf unterschiedliche Schichten des OSI-Modells abzielen:\\

\textbf{Volumetrische Angriffe (Schicht 3/4):} Diese Angriffe zielen darauf ab, die gesamte zur Verfügung stehende Netzwerkbandbreite der Internetanbindung des Ziels zu sättigen. Eine Flut von Paketen wird an ein Ziel gerichtet. Ein klassisches Beispiel ist der UDP-Flood, bei dem eine riesige Menge an UDP-Paketen an zufällige Ports des Zielsystems gesendet wird. Da UDP ein verbindungsloses Protokoll ist, muss der Server für jedes eingehende Paket prüfen, ob eine Anwendung auf dem Port lauscht, und eine ICMP-Antwort "Destination Unreachable" generieren, was seine Ressourcen bindet. Ein ICMP-Flood (oder Ping-Flood) funktioniert nach einem ähnlichen Prinzip, indem er das Ziel mit ICMP-Echo-Request-Paketen überflutet und es zwingt, eine ebenso große Anzahl von Echo-Reply-Paketen zu senden. Das primäre Ziel ist hier die reine Masse an Traffic. \\

\textbf{Protokoll-Angriffe (Schicht 4):} Diese Angriffe nutzen Schwachstellen in der Implementierung von Netzwerkprotokollen aus, um die Ressourcen von Netzwerkkomponenten wie Firewalls oder Load Balancern zu erschöpfen. Der bekannteste Vertreter ist der TCP-SYN-Flood. Er missbraucht den drei-Wege-Handshake von TCP (SYN, SYN-ACK, ACK). Der Angreifer sendet eine hohe Zahl von SYN-Paketen, oft von gefälschten Quell-IP-Adressen. Das Zielsystem antwortet mit SYN-ACK und reserviert Ressourcen in seiner Verbindungstabelle (Backlog Queue) für die erwartete ACK-Antwort. Da diese Antwort niemals eintrifft, bleiben die Verbindungen "halboffen", bis die Tabelle voll ist und keine neuen, legitimen Verbindungsanfragen mehr angenommen werden können und überlastet ist. \\

\textbf{Angriffe auf der Anwendungsebene (Schicht 7):} Diese Angriffe sind subtiler und oft schwerer zu erkennen, da sie scheinbar legitimen Traffic imitieren. Sie zielen nicht auf die Netzwerkbandbreite, sondern auf die CPU- und Speicherressourcen des Servers ab. Beispiele hierfür sind das wiederholte Anfordern sehr großer Dateien oder das Ausführen komplexer Suchanfragen über eine Website, die serverseitig aufwändige Datenbankoperationen auslösen. Auch Angriffe auf Login-Schnittstellen durch massenhafte POST-Requests oder das Ausnutzen rechenintensiver API-Endpunkte fallen in diese Kategorie. Da diese Angriffe oft mit einer relativ geringen Datenrate auskommen, können sie unter dem Radar traditioneller, rein volumetrisch arbeitender Schutzsysteme hindurchgehen.\\ \cite{cloudflare1,rfc1} 

Für ein Intrusion Prevention System besteht die Herausforderung darin, die Muster dieser unterschiedlichen Angriffsarten von legitimen Lastspitzen zu unterscheiden und sie präzise zu blockieren, ohne den regulären Nutzerverkehr zu beeinträchtigen.

\subsection{Malware}

Der Begriff Malware, eine Kurzform für "malicious software", dient als Oberbegriff für jegliche Art von Software, die mit der Absicht entwickelt wurde, auf einem Computersystem unerwünschte oder schädliche Aktionen auszuführen, Daten zu entwenden oder unautorisierten Zugriff zu erlangen. Die Ziele sind dabei vielfältig und reichen vom Diebstahl sensibler persönlicher oder geschäftlicher Informationen über die Störung von Betriebsabläufen, bis hin zur vollständigen Übernahme der Kontrolle über ein System, um es beispielsweise als Teil eines Botnetzes für DDoS-Angriffe zu missbrauchen. Die Infektion eines Netzwerks erfolgt oft mit einem mehrstufigen Prozess.\\

 Zuerst muss die Malware auf das Zielsystem gelangen (Zustellung), was meist durch das Öffnen von manipulierten E-Mail-Anhängen (Phishing), das Klicken auf Links zu kompromittierten Websites (Drive-by-Downloads) oder über infizierte externe Speichermedien geschieht. Anschließend wird die Malware durch eine Benutzerinteraktion oder eine Sicherheitslücke aktiviert (Ausführung) und installiert sich fest im System, um einen Neustart zu überdauern (Persistenz). In vielen Fällen kontaktiert die Schadsoftware danach einen externen Command-and-Control-Server des Angreifers, um Anweisungen zu empfangen oder gestohlene Daten zu übermitteln.\\

Je nach Vorgehensweise und primärer Funktion lässt sich Malware in verschiedene Haupttypen klassifizieren, deren Grenzen jedoch zunehmend verschwimmen. Eine grundlegende Unterscheidung ist für das Verständnis der Bedrohungslandschaft unerlässlich.\\

\textbf{Viren:}\\
Klassische Viren sind Programmsegmente, die sich nicht eigenständig ausführen können, sondern eine legitime, ausführbare Wirtsdatei benötigen, um sich zu verbreiten. Sobald der Nutzer diese infizierte Datei startet, wird auch der virale Code aktiviert, der sich dann in weitere Dateien auf dem System oder auf verbundenen Netzlaufwerken kopiert. Die Schadfunktion, die von der reinen Replikation bis zur Zerstörung von Daten reichen kann, wird erst durch diese Nutzerinteraktion ausgelöst.\cite{BSI1} \\

\textbf{Würmer:}\\
Im Gegensatz dazu sind Würmer eigenständige Schadprogramme, die sich aktiv und ohne Zutun des Nutzers über Netzwerke verbreiten. Sie nutzen gezielt Sicherheitslücken in Betriebssystemen oder Anwendungen aus, um sich von einem infizierten System auf andere, verwundbare Systeme zu replizieren. Dieser autonome Verbreitungsmechanismus kann zu einer extrem schnellen, kaskadenartigen Infektionswelle führen, die ganze Unternehmensnetzwerke lahmlegt. Ein prominentes Beispiel hierfür ist der Wurm WannaCry, der 2017 die ''EternalBlue''-Schwachstelle im SMB-Protokoll von Windows-Systemen ausnutzte, um sich weltweit zu verbreiten und auf den infizierten Systemen Ransomware zu installieren.\cite{BSI12} \\

\textbf{Trojaner:}\\
Trojaner, benannt in Anlehnung an das Trojanische Pferd der griechischen Mythologie, tarnen sich als nützliche oder legitime Software, um den Nutzer zur Installation zu bewegen. Sie enthalten jedoch eine versteckte, bösartige Funktion. Nach der Ausführung installieren sie oft eine Hintertür (Backdoor), die Angreifern einen permanenten und unbemerkten Fernzugriff auf das System ermöglicht. Solche Remote Access Trojans (RATs) erlauben es Angreifern, Daten zu stehlen, das System zu überwachen oder es als Teil eines Botnetzes für weitere Angriffe zu missbrauchen.\cite{kasp1}\\

\textbf{Ransomware:}\\
Eine besonders profitable und schädliche Form ist die Ransomware. Diese Malware verschlüsselt die Dateien des Opfers, sodass auf sie nicht mehr zugegriffen werden kann. Anschließend wird eine Lösegeldforderung angezeigt, meist zahlbar in Kryptowährungen, um die Anonymität der Täter zu wahren. In den letzten Jahren hat sich das Vorgehen zur sogenannten "Double Extortion" (doppelte Erpressung) weiterentwickelt. Hierbei werden die Daten vor der Verschlüsselung zusätzlich vom System des Opfers auf Server der Angreifer kopiert. Wird das Lösegeld nicht gezahlt, drohen die Täter nicht nur mit der permanenten Zerstörung der Daten, sondern auch mit deren Veröffentlichung.\cite{ENISA1}\\

\textbf{Spyware:}\\
Davon abzugrenzen ist Spyware, deren Hauptziel es ist, verdeckt Informationen über den Nutzer, dessen Verhalten und die auf dem System gespeicherten Daten zu sammeln. Eine der bekanntesten Unterarten sind Keylogger, die jeden Tastaturanschlag protokollieren. Auf diese Weise können Angreifer sensible Informationen wie Passwörter, Kreditkartennummern oder private Nachrichten mitschneiden und an ihre Server übermitteln.\\

Durch die Analyse von bekannten Malware-Signaturen, das Erkennen von Anomalien im Protokollverfahren und das Blockieren von Verbindungen zu IP-Adressen und Domains können IPS-Systeme diese Bedrohungen erkennen und unterbinden.



\subsection{Exploits}
Der Begriff Exploit bezeichnet ein spezifisches Software-Fragment, einen Datenblock oder eine Befehlssequenz, die gezielt einen Programmierfehler oder eine konzeptionelle Schwachstelle in einem Computersystem oder einer Anwendung ausnutzt. Es ist entscheidend, zwischen der Schwachstelle, bei der es sich um einen passiven, latenten Fehler im Code handelt, und dem Exploit, dem aktiven Werkzeug zur Ausnutzung dieses Fehlers, zu unterscheiden. Das Ziel eines Exploits ist es, erhöhte Rechte zu erlangen (Privilege Escalation), beliebigen Code auf dem Zielsystem auszuführen (Arbitrary Code Execution) oder einen Denial-of-Service-Zustand auszulösen. Dies macht Exploits zu einem primären Vektor für den erstmaligen Einbruch in ein Netzwerk.\cite{wik12}\\

Der Prozess eines Angriffs, der auf der Ausnutzung von Schwachstellen basiert, lässt sich in die sieben Phasen der Cyber Kill Chain unterteilen, die den Ablauf aus der Perspektive des Angreifers strukturieren. Die erste Phase ist die \textbf{Reconnaissance}, in der ein Angreifer Informationen über das Ziel sammelt, um potenzielle Angriffsvektoren zu finden. In der zweiten Phase, der \textbf{Weaponization}, wird ein passender Exploit-Code mit einer Nutzlast (Payload) – dem eigentlichen Schadcode – gekoppelt. Anschließend erfolgt die \textbf{Delivery}, bei der der vorbereitete Exploit an das Zielsystem übermittelt wird, beispielsweise über eine Phishing-Mail oder einen verwundbaren Netzwerkdienst.\\

In der vierten Phase, der \textbf{Exploitation}, wird der Exploit-Code zur Ausführung gebracht und löst den Fehler in der verwundbaren Software aus. Dies ermöglicht die Ausführung des mitgelieferten Payload. Darauf folgt die \textbf{Installation}, in der die Payload Persistenz auf dem System etabliert, um nach einem Neustart noch auf dem Zielsystem vorhanden zu sein. In der sechsten Phase, \textbf{Command and Control} (C2), baut die installierte Schadsoftware eine ausgehende Verbindung zu einem vom Angreifer kontrollierten Server auf. Über diesen Kanal kann der Angreifer das System fernsteuern. In der letzten Phase, \textbf{Actions on Objectives}, verfolgt der Angreifer seine eigentlichen Ziele, wie zum Beispiel den Diebstahl von Daten, die Ausbreitung im internen Netzwerk, oder die Verschlüsselung von Systemen zur Erpressung von Lösegeld.\cite{wik1}\\

\textbf{Zero-Day-Exploit:}\\
Die gefährlichste Kategorie sind Zero-Day-Exploits. Der Begriff „Zero-Day“ leitet sich aus der Perspektive des Softwareherstellers ab: Ab dem Moment, in dem ein Exploit für eine bisher unbekannte Schwachstelle aktiv ausgenutzt wird, hat der Hersteller null Tage Zeit gehabt, einen entsprechenden Sicherheitspatch zu entwickeln und bereitzustellen. Diese Unkenntnis aufseiten der Verteidiger verschafft den Angreifern ein kritisches Zeitfenster, in dem ihre Angriffe mit sehr hoher Wahrscheinlichkeit erfolgreich sind.\\
Die besondere Gefahr von Zero-Day-Exploits liegt in ihrer Fähigkeit, traditionelle, signaturbasierte Schutzmechanismen wie die meisten Antivirenprogramme und Intrusion Prevention Systeme vollständig zu umgehen. Da der Angriff neu und unbekannt ist, existiert keine Signatur, kein Muster und kein Hashwert, nach dem ein solches System suchen könnte. Der Exploit ist für die Verteidigungslinie demnach quasi unsichtbar.\\
Die Abwehr von Zero-Day-Angriffen erfordert daher fortschrittlichere Sicherheitsstrategien, die über die reine Signaturerkennung hinausgehen. Dazu gehören verhaltensbasierte Analyse (Heuristik) und Anomalieerkennung, bei denen ein System nicht nach bekannten Mustern, sondern nach ungewöhnlichem und potenziell bösartigem Verhalten sucht. Eine weitere wichtige Technik ist das Sandboxing, bei dem verdächtiger Code in einer isolierten, virtuellen Umgebung ausgeführt wird, um seine Aktionen zu beobachten, ohne das eigentliche System zu gefährden.

\section{IDS / IPS Grundlagen}
Nach der detaillierten Betrachtung der vielfältigen Bedrohungslandschaft im vorherigen Kapitel, werden nun die Systeme näher beleuchtet, die entwickelt wurden, um Netzwerke vor genannten Angriffen zu schützen. Während klassische Firewalls den Verkehr primär anhand von Adressen und Ports (Schicht 3 und 4 des OSI-Modells) filtern, gehen moderne Schutzmechanismen einen entscheidenden Schritt weiter, indem sie den Inhalt des Datenverkehrs analysieren. Im Zentrum dieser erweiterten Sicherheitsarchitektur stehen Intrusion Detection- und Intrusion Prevention-Systeme (IDS/IPS).

In den folgenden Absätzen werden dazu die theoretischen Grundlagen dieser Technologien erklärt. Es wird die evolutionäre Entwicklung vom rein passiven Erkennen eines Angriffs (Detection) hin zum aktiven Verhindern (Prevention) nachgezeichnet, um die grundlegenden Unterschiede und die Motivation hinter der Entwicklung von IPS zu beleuchten. Anschließend wird das technische Funktionsprinzip der Deep Packet Inspection (DPI) erläutert, das es diesen Systemen überhaupt erst ermöglicht, den Inhalt von Datenpaketen zu analysieren. Abschließend folgt die Vorstellung der verschiedenen Erkennungsmethoden, auf deren Basis ein System die Entscheidung trifft, ob es sich um legitimen oder bösartigen Datenverkehr handelt.

\subsection{Vom passiven Detektieren zum aktiven Verhindern}

Die historischen Vorläufer moderner Angriffserkennungssysteme sind die Intrusion Detection Systeme (IDS). Ihr grundlegendes Funktionsprinzip ist die der passiven Überwachung. Ein IDS wird im Netzwerk so implementiert, dass es eine Kopie des gesamten zu überwachenden Datenverkehrs erhält, ohne selbst Teil des aktiven Datenpfades zu sein. Technisch wird dies meist über einen sogenannten Mirror-Port (auch SPAN-Port genannt) an einem Netzwerk-Switch realisiert, der den gesamten Verkehr eines oder mehrerer anderer Ports auf den Port des IDS spiegelt. Das IDS agiert somit wie ein Beobachter, der den Verkehr analysiert, ihn jedoch nicht direkt beeinflusst. Stellt das System eine potenzielle Bedrohung fest, die einer seiner vordefinierten Signaturen oder Verhaltensregeln entspricht, ist seine primäre Aufgabe das Auslösen eines Alarms. Dieser Alarm kann in Form eines Eintrags in einer Log-Datei, einer Benachrichtigung per E-Mail an einen Systemadministrator oder einer Meldung an ein übergeordnetes Security Information and Event Management (SIEM) System erfolgen. Die Kernfunktion eines IDS ist also die reine Detektion; es beantwortet die Frage: „Passiert hier gerade etwas Bösartiges?“ \cite{Claudia1}.\\

Die entscheidende Schwäche dieses passiven Ansatzes liegt in der systembedingten Latenz zwischen der Erkennung eines Angriffs und der Einleitung einer Gegenmaßnahme. Nachdem ein IDS einen Alarm ausgelöst hat, ist eine Reaktion erforderlich, um den Angriff zu stoppen. Diese Reaktion kann manuell durch einen Administrator erfolgen, der beispielsweise eine neue Regel in der Firewall konfiguriert, oder durch ein nachgeschaltetes, automatisiertes System. In der Zeit, die dieser Prozess unweigerlich in Anspruch nimmt, kann der Angriff sein Ziel bereits erreicht und erheblichen Schaden verursacht haben.\\

Um diese kritische Schutzlücke zu schließen wurden Intrusion Prevention Systeme (IPS) entwickelt. Im Gegensatz zu einem IDS wird ein IPS aktiv und ''in-line'' im Netzwerk platziert und behebt damit dessen grundlegende Schwäche. Damit muss der gesamte Datenverkehr das IPS passieren, um sein Ziel zu erreichen. \\

Diese Positionierung ermöglicht es einem IPS, nicht nur zu erkennen, sondern auch unmittelbar zu handeln. Erkennt es einen Angriff, kann es eine Reihe von vordefinierten, automatisierten Aktionen durchführen. Die Palette dieser Reaktionsmöglichkeiten ist breit und reicht von dem einfachen Blockieren oder Verwerfen (Block/Drop) einzelner bösartiger Datenpakete, über das aktive Zurücksetzen von TCP-Verbindungen (Reset), bis hin zur reinen Alarmierung (Alert), bei der der Verkehr zu Testzwecken bewusst durchgelassen wird.\cite{Suricata1, NIST1}\\

Darüber hinaus bieten viele Systeme erweiterte, oftmals herstellerabhängige Reaktionsmöglichkeiten. Dazu zählt beispielsweise die temporäre Sperrung der angreifenden IP-Adresse oder die Umleitung des Angriffs auf ein Ködersystem zur weiteren Analyse (Quarantäne). Die genaue Auswahl der verfügbaren Aktionen sind ein wesentliches Merkmal des jeweiligen Produkts. Welche dieser Funktionen von den in dieser Arbeit untersuchten Systemen, FortiGate und OPNsense, im Detail unterstützt werden, wird im Rahmen ihrer jeweiligen Vorstellung in Kapitel 2.3 erläutert.
\subsection{Funktionsprinzip: Deep Packet Inspection (DPI)}
\subsection{Zentrale Erkennungsmethoden}

\section{Vorstellung der Systeme}
\subsection{Der kommerzielle Vertreter: FortiGate}
\subsection{Die Open-Source-Alternative: OPNsense}
\subsection{Synoptische Gegenüberstellung}