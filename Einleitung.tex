\chapter{Einleitung}

\section{Motivation und Problemstellung}
Die Digitalisierung durchdringt unaufhaltsam alle Bereiche von Wirtschaft und Gesellschaft. Während sie Effizienzgewinne, neue Geschäftsmodelle und eine globale Vernetzung ermöglicht, schafft sie gleichzeitig eine ebenso wachsende wie komplexe Angriffsfläche für Kriminalität. Die Bedrohungslage im Cyberraum hat sich in den letzten Jahren dramatisch verschärft und ist von einer Randnotiz für IT-Spezialisten zu einer strategischen Herausforderung für Unternehmen jeder Größenordnung geworden. Das Bundesamt für Sicherheit in der Informationstechnik (BSI) bewertet die Lage in seinem jährlichen Bericht als „angespannt bis besorgniserregend“ und verweist auf eine zunehmende Professionalisierung und Arbeitsteilung in der cyberkriminellen Schattenwirtschaft . \cite{BSI1}
\\\\
Die Angriffsvektoren sind vielfältig und reichen von Denial-of-Service-Angriffen (DoS), die gezielt die Nichtverfügbarkeit von Diensten herbeiführen, über die Infiltration von Netzwerken mittels Schadsoftware (Malware) wie Viren und Trojanern, bis hin zur aktiven Ausnutzung von Software-Schwachstellen durch Exploits. Insbesondere Ransomware-Angriffe, bei denen Daten verschlüsselt und nur gegen Lösegeld wieder freigegeben werden, haben sich zu einer der größten Bedrohungen für die deutsche Wirtschaft entwickelt. Laut einer Studie des Digitalverbands Bitkom entstanden hierdurch allein im Jahr 2024 Schäden in einer Höhe von 266,6 Milliarden Euro, wobei ein Großteil der Angriffe auf kleine und mittelständische Unternehmen (KMU) zielte. \cite{bitkom1}
\\\\
Gerade KMU stehen vor einer besonderen Herausforderung: Sie verfügen oft nicht über die gleichen finanziellen und personellen Ressourcen wie Großkonzerne, um sich umfassend zu schützen. Gleichzeitig sind sie aufgrund ihrer Rolle in Lieferketten und als Träger der regionalen Wirtschaft ein attraktives Ziel. Der traditionelle Schutz des Unternehmensnetzwerks durch eine reine Paketfilter-Firewall, die den Verkehr nur anhand von Ports und IP-Adressen kontrolliert, ist angesichts der modernen Bedrohungslage nicht mehr ausreichend. Angriffe finden heute oft auf der Anwendungsebene statt und verstecken sich im legitimen Datenverkehr, beispielsweise über den standardmäßigen Web-Port 443.
\\\\
An dieser Stelle setzen Intrusion Prevention Systeme (IPS) an. Als Weiterentwicklung von Intrusion Detection Systemen (IDS), die Angriffe nur erkennen und melden, sind IPS in der Lage, bösartigen Datenverkehr aktiv zu analysieren, zu klassifizieren und in Echtzeit zu blockieren, bevor er Schaden im internen Netzwerk anrichten kann. Diese Systeme agieren als wachsame "Torwächter", die den Inhalt der Datenpakete tiefgehend inspizieren (Deep Packet Inspection) und mit bekannten Angriffsmustern (Signaturen) oder anormalem Verhalten abgleichen.
\\\\
Für Unternehmen, die eine solche Schutztechnologie implementieren möchten, stellt sich eine grundlegende strategische Frage: Soll auf eine kommerzielle „All-in-One“-Lösung eines etablierten Herstellers gesetzt werden, oder kann eine flexiblere und potenziell kostengünstigere Open-Source-Lösung einen vergleichbaren Schutz bieten? Kommerzielle Produkte, oft als Next-Generation Firewalls (NGFW) vermarktet, versprechen eine hohe Integration, professionellen Support und eine einfache Verwaltung aus einer Hand. Demgegenüber stehen Open-Source-Alternativen, die durch Transparenz, Anpassbarkeit und den Wegfall von Lizenzgebühren überzeugen, jedoch oft ein höheres Maß an technischem Know-how in der Implementierung und Wartung erfordern.\\\\

Zusätzlich zu diesen externen Bedrohungen hat sich die Erkenntnis durchgesetzt, dass das interne Netzwerk (LAN) nicht per se als vertrauenswürdig gelten kann. Angriffe können ebenso von innen heraus erfolgen, sei es durch kompromittierte Endgeräte, die bereits Teil des Netzwerks sind, oder durch böswillige Insider. Dieses Paradigma, bekannt als Zero Trust, geht davon aus, dass Bedrohungen überall lauern können und somit auch der Datenverkehr innerhalb des Netzwerks einer Überprüfung bedarf. Eine zentrale Aufgabe moderner IPS ist es daher nicht mehr nur, die Grenze zum Internet zu schützen, sondern auch eine Segmentierung und Überwachung des internen Datenverkehrs zu ermöglichen.
\section{Stand der Technik und Forschungslücke}

Die vorliegende Arbeit positioniert sich exakt in diesem Spannungsfeld. Sie zielt darauf ab, den in der Praxis relevanten Entscheidungsprozess zwischen einem kommerziellen und einem Open-Source-System durch einen systematischen, technischen und praktischen Vergleich zu objektivieren.
\\\\
Als Repräsentant für die kommerzielle Welt wird eine FortiGate 70F der Firma Fortinet herangezogen. Fortinet ist einer der weltweiten Marktführer im Bereich der Netzwerksicherheit. Die FortiGate-Serie integriert Firewall-, VPN-, und IPS-Funktionen in einer einzigen Hardware-Appliance, die durch den Einsatz spezialisierter Security Processors (SPs) eine hohe Performance verspricht. Die 
\\\\
Als Gegenstück wird OPNsense untersucht, eine weit verbreitete und aktiv weiterentwickelte Open-Source-Firewall-Distribution, die auf dem robusten Betriebssystem FreeBSD basiert. OPNsense kann auf Standard-Hardware betrieben werden und integriert für die IPS-Funktionalität das leistungsfähige Open-Source-Engine Suricata. Suricata ist ein hochperformantes IDS/IPS, das von der Open Information Security Foundation (OISF) entwickelt wird. Im Gegensatz zum kommerziellen Ansatz ist der Anwender hier selbst dafür verantwortlich, die Hardware zu dimensionieren und die Regelwerke (Signaturen) aus verschiedenen freien oder auch kostenpflichtigen Quellen zu beziehen und zu verwalten.
\\\\
Während es zahlreiche Marketing-Vergleiche und allgemeine Gegenüberstellungen von Vor- und Nachteilen beider Ansätze gibt, fehlt es an detaillierten, praxisorientierten und reproduzierbaren Vergleichsstudien. Oft bleiben solche Analysen an der Oberfläche und vergleichen lediglich Feature-Listen, ohne die tatsächliche Schutzwirkung und die betrieblichen Aspekte in einer kontrollierten Umgebung zu testen. Diese Arbeit schließt diese Lücke, indem sie nicht nur die theoretischen Konzepte, sondern die konkrete Implementierung und Leistungsfähigkeit von FortiGate und OPNsense in einem eigens aufgebauten Testlabor direkt gegenüberstellt.

\section{Zielsetzung und Forschungsfragen}

Das primäre Ziel dieser Bachelorarbeit ist es, eine fundierte und objektive Entscheidungsgrundlage vor allem für KMU zu schaffen, die vor der Wahl eines geeigneten IPS stehen. Es soll ermittelt werden, inwieweit die Open-Source-Lösung OPNsense eine technisch ebenbürtige und praktikable Alternative zum kommerziellen Marktführer FortiGate darstellt.\\

\textbf{Um dieses Ziel zu erreichen, wird die Untersuchung von folgender zentraler Forschungsfrage geleitet:}\\

Inwieweit kann das Open-Source-System OPNsense in Bezug auf Schutzwirkung, Performance und Verwaltungsaufwand eine effektive Alternative zur kommerziellen Next-Generation-Firewall FortiGate für den Einsatz in kleinen und mittelständischen Unternehmen darstellen?\\

\textbf{Zur systematischen Beantwortung dieser Hauptfrage werden die folgenden untergeordneten Forschungsfragen untersucht:}\\

Worin liegen die konzeptionellen und architektonischen Unterschiede bei der Implementierung der IPS-Funktionalität in der FortiGate-Appliance und dem OPNsense-System mit Suricata?\\

Wie effektiv erkennen und blockieren beide Systeme eine Reihe definierter, praxisnaher Angriffe aus den Kategorien Malware, Exploits und Denial-of-Service in einer kontrollierten Testumgebung?\\

Welche qualitativen Unterschiede bestehen hinsichtlich des Konfigurationsaufwands, der Benutzerfreundlichkeit der Verwaltungsoberflächen und der Qualität der Protokollierungs- und Reporting-Funktionen?\\

Welche quantifizierbaren Auswirkungen hat die Aktivierung der IPS-Funktionen auf den Netzwerkdurchsatz und die Latenz bei beiden Systemen?

\section{Aufbau der Arbeit}

Die Struktur der vorliegenden Arbeit orientiert sich konsequent an der Beantwortung der formulierten Forschungsfragen.\\

Das erste Kapitel legt die theoretischen Grundlagen. Es wird ein Überblick über die aktuelle Bedrohungslage gegeben und die Funktionsweise sowie die technologischen Konzepte von Intrusion Detection- und Prevention Systemen erläutert. Anschließend werden die beiden zu vergleichenden Systeme, FortiGate und OPNsense, mit ihren jeweiligen Architekturen und Besonderheiten vorgestellt.\\

Im zweiten Kapitel wird der Aufbau der Testumgebung detailliert beschrieben. Dies umfasst die Auswahl der Hardware- und Softwarekomponenten, die entworfene Netzwerkarchitektur sowie die grundlegende Konfiguration der beiden IPS-Systeme, um eine transparente und nachvollziehbare Testbasis zu schaffen.\\

Das dritte Kapitel widmet sich dem Testing. Hier werden die konkreten Testfälle, die zur Überprüfung der Schutzwirkung und Performance dienen, definiert und deren Durchführung systematisch protokolliert.\\

Die im Praxisteil gewonnenen Daten werden im vierten Kapitel ausgewertet. Zunächst werden die Testergebnisse analysiert und anschließend die beiden Systeme anhand der zuvor definierten Kriterien wie Erkennungsrate, Performance und Benutzerfreundlichkeit verglichen. \\

Das fünfte und letzte Kapitel fasst die Erkenntnisse in einem Fazit zusammen, beantwortet die Forschungsfragen und gibt eine abschließende Bewertung sowie eine Handlungsempfehlung ab. 