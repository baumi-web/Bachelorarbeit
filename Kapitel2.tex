\chapter{Aufbau der Testumgebung}
\section{Physische und Virtuelle Komponenten}
\subsection{Test-Hardware}
\subsection{Virtualisierungssoftware}

Als Fundament für die Virtualisierung der Client-Systeme, die das LAN-Segment der Testumgebung abbilden, wurde die Open-Source-Plattform Proxmox Virtual Environment (VE) in der Version 8.2 gewählt. Die Entscheidung für einen dedizierten Bare-Metal-Hypervisor anstelle einer Desktop-Virtualisierungslösung wie VirtualBox wurde getroffen, um eine höhere Performance durch den direkten, ressourcenschonenden Zugriff auf die Hardware des Test-Laptops zu gewährleisten. Proxmox VE bot zudem die notwendige, granulare Kontrolle über die virtuelle Vernetzung und eine integrierte Snapshot-Funktionalität. Dieses Feature war für die Durchführung der Tests essenziell, da sie es ermöglicht, die Testsysteme vor jedem Testdurchlauf in einen definierten, sauberen Ausgangszustand zurückzusetzen und so für reproduzierbare Ergebnisse zu sorgen.\\\\

Die Installation erfolgte über das offizielle ISO-Abbild direkt auf dem dafür vorgesehenen Test-Laptop, der ausschließlich als Host für die Client-Systeme dient. Hierfür wurde zunächst mit dem Werkzeug "Rufus" ein bootfähiger USB-Stick erstellt. Der Installationsprozess selbst gestaltet sich unkompliziert, erforderte jedoch bei der Netzwerkkonfiguration einzelne manuelle Anpassungen. Da der Laptop als Server im LAN-Segment hinter der physischen Firewall agieren sollt, muss eine statische IP-Adresse, Subnetzmaske, Gateway (die LAN-IP-Adresse der Firewall), sowie der DNS-Server manuell in der Installationsroutine hinterlegt werden.\\\\

Ein zentrales Konzept für den Aufbau der Testumgebung in Proxmox ist dabei die Verwendung von virtuellen Bridges. Für den in dieser Arbeit realisierten Aufbau wurde eine einzige virtuelle Bridge (vmbr0) konfiguriert, die direkt an die physische Netzwerkkarte des Proxmox-Laptops gekoppelt ist. Das physische LAN-Kabel verbindet diese Netzwerkkarte mit dem LAN-Port der zu testenden, physischen Firewall. Alle in Proxmox erstellten Client-VMs werden dann mit ihren virtuellen Netzwerkkarten an diese Bridge angeschlossen, wodurch sie zusammen ein virtuelles LAN-Segment bilden. Dieses Segment ist physisch direkt und ausschließlich mit einem geschützten LAN-Port der Firewall verbunden.\\

\textbf{Aufgetretene Probleme:}\\
Bei der Installation  traten initial Konnektivitätsprobleme auf. Anschließend war die webbasierte Verwaltungsoberfläche von einem anderen Rechner im Netzwerk nicht erreichbar, obwohl die Netzwerkinformationen im Installer korrekt eingegeben wurden. Eine Analyse über die Kommandozeile des Servers zeigte, dass die Netzwerkkonfigurationsdatei ''\textit{/etc/network/interfaces}'' nicht korrekt geschrieben wurde. Dieses Problem konnte durch eine manuelle Bearbeitung der Datei direkt auf der Server-Konsole und einen anschließenden Neustart des Netzwerk-Dienstes mittels ''\textit{systemctl restart networking.service}'' behoben werden. Erst danach war der Proxmox-Host unter der korrekten statischen IP-Adresse erreichbar und die weitere Konfiguration konnte über die Weboberfläche erfolgen.
\subsection{Virtuelle Maschinen (VMs)}

\section{Logische Netzwerkarchitektur}

Für die Durchführung der Tests wurde eine logische Netzwerkarchitektur entworfen, die eine realitätsnahe Topologie nachbildet. Das grundlegende Design basiert auf einer klaren Segmentierung in zwei separate Netzwerkzonen: ein unsicheres Wide Area Network (WAN), das das Internet simuliert, und ein durch eine Firewall geschütztes Local Area Network (LAN), das das interne Unternehmensnetzwerk darstellt. Die zu testende physische Firewall fungiert als zentraler Kopplungspunkt und einziges Gateway zwischen diesen beiden Zonen.\\\\

Das WAN-Segment wurde als Angriffsnetzwerk konzipiert. In diesem Netz befindet sich ausschließlich der dedizierte Angreifer-Laptop mit dem Betriebssystem Kali Linux. Diesem Netzwerk wurde der IP-Adressbereich 192.168.107.0/24 zugewiesen. Der WAN-Port der Firewall wurde mit einer statischen IP-Adresse aus diesem Bereich konfiguriert (192.168.107.1) und dient als Gateway für den Angreifer-PC.\\\\

Das LAN-Segment stellt die zu schützende Umgebung dar. Es besteht aus dem Proxmox-Server, dessen physische Netzwerkkarte direkt mit dem LAN-Port der Firewall verbunden ist. Alle für die Tests benötigten Client-Systeme (Windows 11, Ubuntu, Metasploitable) werden als virtuelle Maschinen innerhalb dieses Proxmox-Hosts betrieben. Dem LAN-Segment wurde der separate IP-Adressbereich 192.168.101.0/24 zugeteilt. Der LAN-Port der Firewall agiert hier als Standardgateway (192.168.101.1) für alle VMs und versorgt diese mittels eines auf der Firewall konfigurierten DHCP-Servers automatisch mit IP-Adressen.\\\\

Diese Architektur stellt sicher, dass jeder Angriffsversuch, der vom Kali-Linux-System im WAN ausgeht und auf einen der virtuellen Clients im LAN zielt, zwangsläufig die physische Firewall passieren muss. Der Datenverkehr wird am WAN-Port empfangen, von den Firewall-Richtlinien sowie der Intrusion-Prevention-Engine verarbeitet und erst dann, falls die Regeln es erlauben, an den LAN-Port und somit an das Zielsystem weitergeleitet. Dadurch ist die Firewall der einzige und obligatorische Inspektionspunkt für den gesamten relevanten Testverkehr, was die Grundvoraussetzung für einen validen Vergleich der beiden Systeme darstellt.

\section{Grundkonfiguration der IPS-Systeme}
\subsection{Konfiguration der FortiGate 70F}
\subsection{Konfiguration von OPNsense}