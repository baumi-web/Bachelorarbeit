\chapter{Aufbau der Testumgebung}
\section{Physische und Virtuelle Komponenten}
\subsection{Test-Hardware}
\textbf{Hardware für OpnSense:}\\
Als Open-Source-Gegenstück zur kommerziellen Appliance wurde bewusst auf spezialisierte Sicherheits-Hardware verzichtet. Stattdessen kommt der Open-Source-Philosophie folgend, ein für solche Zwecke in der Community weit verbreiteter, kompakter Embedded-PC vom Typ PC Engines APU4, der gern als Router oder Firewall genutzt wird, zum Einsatz. Die Auswahl dieses Geräts aus dem bereits vorhandenen Bestand ist dabei fundamentaler Teil der vergleichenden Methodik. Sie simuliert ein praxisnahes Szenario für eine Organisation mit begrenztem Budget und unterstreicht den starken konzeptionellen Gegensatz zur FortiGate: Hier steht die Flexibilität und Unabhängigkeit von spezifischen Herstellern, eine offene Software auf kostengünstiger Standard-Hardware zu betreiben, dem geschlossenen Ökosystem einer herstellerspezifischen, hochintegrierten und kostenintensiven Appliance gegenüber.\\

Die Hardware-Analyse mittels \textit{dmidecode} bestätigt die Verwendung eines für solche Systeme typischen coreboot-BIOS. Das Gerät ist mit vier dedizierten Intel-Gigabit-Netzwerkschnittstellen, 4 GB RAM und einer 64 GB SSD ausgestattet. (siehe Abbildung~\ref{fig:hardware-apu4})\\

Da diese Art von Hardware als ''headless'' System konzipiert ist und über keine eigene Grafikschnittstelle verfügt, musste die Installation und Erstkonfiguration vollständig über eine serielle Konsole erfolgen.

Zunächst wurde das offizielle serielle Image von OPNsense heruntergeladen und ein bootfähiger USB-Stick erstellt. Für den Zugriff auf die Konsole wurde der Test-Laptop über ein Nullmodem-Seriell-Kabel und einen USB-Adapter mit dem Konsolen-Port des APU4-Boards verbunden. Mittels des Terminal-Programms Putty wurde dann eine serielle Verbindung mit der korrekten Baudrate (115200) hergestellt.\\

Nach dem Starten des APU4-Boards vom vorbereiteten USB-Stick konnte der gesamte Installationsprozess über dieses Terminal-Fenster gesteuert werden. Nach dem initialen Login mit den Standard-Anmeldeinformationen wurde der textbasierte OPNsense-Installer gestartet, der das System auf die interne SSD installierte. Der letzte Schritt der Grundinstallation war die Zuweisung der physischen Netzwerkports \textit{igb0} und \textit{igb1} zu den logischen OPNsense-Schnittstellen (WAN und LAN), was ebenfalls über das Konsolenmenü erfolgte. Erst nach Abschluss dieser auf der Kommandozeile basierenden Erst-Inbetriebnahme war das System über seine LAN-Schnittstelle erreichbar.
Nun konnte man sich an den Netzwerkport für LAN anstecken und bekam per DHCP eine IP-Adresse. Die weiterführende Konfiguration konnte über die webbasierte grafische Benutzeroberfläche vorgenommen werden.\\

\textbf{Hardware für Virtualisierung:}\\
Als Virtualisierungs-Host, der die zu schützenden Client-Systeme im LAN-Segment beherbergt, dient ein kompakter Mini-PC vom Typ HP EliteDesk 800 G4 DM 35W. Das System ist mit einer Intel(R) Core(TM) i5-8500T CPU, mit einem 16 GB DDR4-Arbeitsspeicher ausgestattet. Als primärer Datenspeicher für die virtuellen Maschinen wurde eine 256 GB NVMe SSD gewählt.(siehe Abbildung~\ref{fig:hardware-prox})\\

Die Entscheidung für eine SSD anstelle einer mechanischen Festplatte (HDD) war für die Performance der Testumgebung entscheidend. In einer Virtualisierungsumgebung greifen mehrere VMs gleichzeitig auf den Speicher zu, was zu einer hohen Anzahl paralleler Lese- und Schreibvorgänge führt. Während eine HDD diese Anfragen aufgrund ihres mechanischen Aufbaus mit einem beweglichen Schreib-/Lesekopf nur sequenziell, also Schritt für Schritt, abarbeiten kann, ist eine SSD in der Lage, Tausende dieser Anfragen parallel zu bedienen. Dieser Vorteil verhindert einen Input/Output-Flaschenhals, der für die nachfolgenden Tests zu ungenauen Ergebnissen führen könnte.\\

\textbf{Angreifer-Systems (Kali Linux):}\\
Als Angreifer-System, das im ungesicherten WAN-Segment der Testumgebung agiert, wurde ein dedizierter Laptop vom Typ DELL Latitude 5520 verwendet. Das Gerät ist mit einem Intel(R) Core(TM) i7 Prozessor, 16 GB RAM und einer 500 GB SSD ausgestattet. Diese Hardware-Ressourcen wurden als mehr als ausreichend bewertet, um die für die Tests notwendigen Analyse- und Angriffswerkzeuge performant auszuführen.\\

Die Installation des Betriebssystems erfolgte mittels der offiziellen Installer ISO-Datei von Kali Linux (Version 2025.2). Um den Laptop von diesem Abbild zu starten, wurde mit dem Programm ''Rufus'' ein bootfähiger USB-Stick erstellt. Während des Installationsprozesses wurde die gesamte 500 GB SSD für Kali Linux partitioniert und ein Standardbenutzer für die Durchführung der Tests angelegt.\\

Ein entscheidender Schritt war die Netzwerkkonfiguration. Da sich der Angreifer-Laptop in einem isolierten, physischen Netzwerksegment (WAN\_KALI) befindet, das nur mit dem WAN-Port der zu testenden Firewall verbunden ist, musste eine statische IP-Konfiguration vorgenommen werden. Dies wurde über die Kommandozeilen-Schnittstelle \textit{nmcli} realisiert. Dem System wurde die IP-Adresse 192.168.107.40 /24 zugewiesen. Als Gateway wurde die IP-Adresse des WAN\_KALI-Interfaces der Firewall (192.168.107.1) eingetragen, um eine korrekte Routing-Beziehung herzustellen.\\

Nach der erfolgreichen Grundinstallation wurde das System zunächst auf den neuesten Stand gebracht. Hierfür wurden über das Terminal die Befehle \textit{sudo apt update} und \textit{sudo apt full-upgrade -y} ausgeführt. Um sicherzustellen, dass alle für die geplanten Tests notwendigen Werkzeuge zur Verfügung stehen, wurde anschließend das Metapackage kali-linux-default installiert. Dieser Befehl (\textit{sudo apt install -y kali-linux-default}) rüstet das System mit der Standard-Sammlung an Penetration-Testing-Tools aus, die unter anderem das Metasploit Framework, den Portscanner Nmap sowie den Paketgenerator hping3 umfasst.\\


\subsection{Virtualisierungssoftware}

Als Fundament für die Virtualisierung der Client-Systeme, die das LAN-Segment der Testumgebung abbilden, wurde die Open-Source-Plattform Proxmox Virtual Environment (VE) in der Version 8.2 gewählt. Die Entscheidung für einen dedizierten Bare-Metal-Hypervisor anstelle einer Desktop-Virtualisierungslösung wie VirtualBox wurde getroffen, um eine höhere Performance durch den direkten, ressourcenschonenden Zugriff auf die Hardware des Test-Laptops zu gewährleisten. Proxmox VE bot zudem die notwendige, granulare Kontrolle über die virtuelle Vernetzung und eine integrierte Snapshot-Funktionalität. Dieses Feature war für die Durchführung der Tests essenziell, da sie es ermöglicht, die Testsysteme vor jedem Testdurchlauf in einen definierten, sauberen Ausgangszustand zurückzusetzen und so für reproduzierbare Ergebnisse zu sorgen.\\

Die Installation erfolgte über das offizielle ISO-Abbild direkt auf dem dafür vorgesehenen Test-Laptop, der ausschließlich als Host für die Client-Systeme dient. Hierfür wurde zunächst ein bootfähiger USB-Stick erstellt. Der Installationsprozess selbst gestaltet sich unkompliziert, erforderte jedoch bei der Netzwerkkonfiguration einzelne manuelle Anpassungen. Da der Laptop als Server im LAN-Segment hinter der physischen Firewall agieren sollt, muss eine statische IP-Adresse, Subnetzmaske, Gateway (die LAN-IP-Adresse der Firewall), sowie der DNS-Server manuell in der Installationsroutine hinterlegt werden.\\

\textbf{Aufgetretenes Problem:}\\
Bei der Installation  traten initial Konnektivitätsprobleme auf. Anschließend war die webbasierte Verwaltungsoberfläche von einem anderen Rechner im Netzwerk nicht erreichbar, obwohl die Netzwerkinformationen im Installer korrekt eingegeben wurden. Eine Analyse über die Kommandozeile des Servers zeigte, dass die Netzwerkkonfigurationsdatei ''\textit{/etc/network/interfaces}'' nicht korrekt geschrieben wurde. Dieses Problem konnte durch eine manuelle Bearbeitung der Datei direkt auf der Server-Konsole und einen anschließenden Neustart des Netzwerk-Dienstes mittels ''\textit{systemctl restart networking.service}'' behoben werden. Erst danach war der Proxmox-Host unter der korrekten statischen IP-Adresse erreichbar und die weitere Konfiguration konnte über die Weboberfläche erfolgen.\\

Ein zentrales Konzept für den Aufbau der Testumgebung in Proxmox ist dabei die Verwendung von virtuellen Bridges. Für den in dieser Arbeit realisierten Aufbau wurde eine einzige virtuelle Bridge (vmbr0) konfiguriert, die direkt an die physische Netzwerkkarte des Proxmox-Laptops gekoppelt ist. Das physische LAN-Kabel verbindet diese Netzwerkkarte mit dem LAN-Port der zu testenden, physischen Firewall. Alle in Proxmox erstellten Client-VMs werden dann mit ihren virtuellen Netzwerkkarten an diese Bridge angeschlossen, wodurch sie zusammen ein virtuelles LAN-Segment bilden. Dieses Segment ist physisch direkt und ausschließlich mit einem geschützten LAN-Port der Firewall verbunden.\\


\subsection{Virtuelle Maschinen (VMs)}

\section{Logische Netzwerkarchitektur}

Für die Durchführung der Tests wurde eine logische Netzwerkarchitektur entworfen, die eine realitätsnahe Topologie nachbildet. Das grundlegende Design basiert auf einer klaren Segmentierung in zwei separate Netzwerkzonen: ein unsicheres Wide Area Network (WAN), das das Internet simuliert, und ein durch eine Firewall geschütztes Local Area Network (LAN), das das interne Unternehmensnetzwerk darstellt. Die zu testende physische Firewall fungiert als zentraler Kopplungspunkt und einziges Gateway zwischen diesen beiden Zonen.\\

Das WAN-Segment wurde als Angriffsnetzwerk konzipiert. In diesem Netz befindet sich ausschließlich der dedizierte Angreifer-Laptop mit dem Betriebssystem Kali Linux. Diesem Netzwerk wurde der IP-Adressbereich 192.168.107.0/24 zugewiesen. Der WAN-Port der Firewall wurde mit einer statischen IP-Adresse aus diesem Bereich konfiguriert (192.168.107.1) und dient als Gateway für den Angreifer-PC.\\

Das LAN-Segment stellt die zu schützende Umgebung dar. Es besteht aus dem Proxmox-Server, dessen physische Netzwerkkarte direkt mit dem LAN-Port der Firewall verbunden ist. Alle für die Tests benötigten Client-Systeme (Windows 11, Ubuntu, Metasploitable) werden als virtuelle Maschinen innerhalb dieses Proxmox-Hosts betrieben. Dem LAN-Segment wurde der separate IP-Adressbereich 192.168.101.0/24 zugeteilt. Der LAN-Port der Firewall agiert hier als Standardgateway (192.168.101.1) für alle VMs und versorgt diese mittels eines auf der Firewall konfigurierten DHCP-Servers automatisch mit IP-Adressen.\\

Diese Architektur stellt sicher, dass jeder Angriffsversuch, der vom Kali-Linux-System im WAN ausgeht und auf einen der virtuellen Clients im LAN zielt, zwangsläufig die physische Firewall passieren muss. Der Datenverkehr wird am WAN-Port empfangen, von den Firewall-Richtlinien sowie der Intrusion-Prevention-Engine verarbeitet und erst dann, falls die Regeln es erlauben, an den LAN-Port und somit an das Zielsystem weitergeleitet. Dadurch ist die Firewall der einzige und obligatorische Inspektionspunkt für den gesamten relevanten Testverkehr, was die Grundvoraussetzung für einen validen Vergleich der beiden Systeme darstellt.

\section{Grundkonfiguration der IPS-Systeme}
\subsection{Konfiguration der FortiGate 70F}
\subsection{Konfiguration von OPNsense}